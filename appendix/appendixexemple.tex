\chapter*{APPENDICES}
\addcontentsline{toc}{chapter}{APPENDICES}

\begingroup\let\clearpage\relax
\chapter{Guide to Generate an Animation}
\label{chap:appendix_arb}

\endgroup
\begin{enumerate}
	\item Organise a PC, laptop or server that is powerful enough to render and runs linux.
	\item Install blender, git, ffmpeg and make on it. 
	\begin{itemize}
		\item On Ubuntu or Debian this is done with 
		
		\texttt{apt install blender make git ffmpeg}
		\item On Arch Linux this is done with 
		
		\texttt{pacman -S blender make git ffmpeg}
	\end{itemize}
	\item Clone the git repository for where you want to work
	
	\texttt{ git clone https://github.com/timoll/eye-generator}
	
	In case you want to contribute to the project, fork the project on github and clone your repository. Once you made meaningful changes you can push them to your repository and create a pull request.
	\item Change directory into the cloned project
	
	\texttt{cd eye-generator}
	\item There are already a few animations stored in the repository. Open a .json file for reference.
	
	\texttt{gedit animation.json}
	
	You can adjust the values manualy, write a script that generates the animation you want or modify it on \url{http://jsoneditoronline.org/} 
	\begin{labeling}{Right/Left Eye Keyframes   }
		\item [Eye Position Left/Right] Position of the Left/Right Eye in centimeters. 
		\item [Diameter] Diameter of the eye in millimetres
		\item [Last Frame] The last frame that will be rendered
		\item [Right/Left Eye Keyframes] Keyframes of the Left/Right eye. Frames in between are interpolated
		
		\begin{labeling}{Rotationnn}
			\item[Frame] When the eye has this position
			\item[Rotation] The direction the eye is looking \{-90, 0, 0\} is forward. Change x for up/down and z for left/right
		\end{labeling}
		\item [Glasses Keyframes] Keyframes for the glasses
		\begin{labeling}{Rotationnn}
			\item[Frame] When the eye has this position
			\item[Position] Position of the Glasses \{0 , 0, 0\} is a good start
			\item[Rotation] The direction the glasses is pointing \{-90, 0, 0\} is forward. 
		\end{labeling}
	\end{labeling}
	\item Save the new json file in the same folder as the rest of the project
	\item Export your file so make knows it
	
	\texttt{export ANIMATION\_JSON=myanimation.json}
	\item Create the different blend files with your animation
	
	\texttt{make}
	\item (optional) Verify that your animation is right in blender
	
	\texttt{blender leftup.blend}
	
	In the bottom left corner select "Timeline" and play the animation. You can zoom out or in by scrolling and move around by pressing the middle mouse button.
	\item Start the render, make sure this is done on the server if you have access to one. 
	
	\texttt{make render}
	It will start the render detached so you can continue to use the command line. If you want to abort the render you need to stop blender
	
	\texttt{pkill blender}
	
	Note: this will kill every instance of blender so make sure you save open blend files first.
	\item Wait, this process may take some time, especially if the pc is not that fast.
	\item Once all frames are rendered you can inspect them in their folders. The blender output is saved as log in log/renderlu.log or similar for each camera.
	
	The position and rotation of the glasses and the rotation of eyes for each frame are logged also in the log folder
	\item(optional) You might want to generate a movie files from the pictures. For the eye cameras just run the script \texttt{./genffmpeg}. It will generate video files in the folder videos.
	\item Cleaning up. Once you have saved the generated data that you need. Clean up the folder by running
	
	\texttt{make clean}
	
	To delete all the blend files that are not needed and
	
	\texttt{make clean\_render}
	
	To delete all files that where generated by the render. 
	
	

	
	
	
	
\end{enumerate}