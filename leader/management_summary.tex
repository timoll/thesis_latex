\chapter*{Management Summary}
\label{chap:managementSummary}

Eye-tracking offers important insights about hand-eye-coordination and visual search techniques of an athlete. An eye-tracking system for sports needs a high accuracy, portability and high temporal resolution.

Such a device is under development at the Institute for Human Centered Engineering. The system uses two cameras per eye to capture infrared images. The images should be used to determine the gaze direction of the athlete. An important component for that is the digital recognition of the pupil. This is currently not implemented satisfactory.

Until now, the detection of the pupil was achieved with a hardly tested algorithm. This algorithm should be improved so that it is more stable and accomplish a better accuracy. Additionally, the algorithm needs to run on the portable hardware of the eye-tracking system.

Test data is required to compare the accuracy of the algorithms. This data needs to consist of known actual values. To achieve that, an animated and accurate model of the eye and the eye-tracking system should be created. This model should be flexible so that pictures for different scenarios can be generated. The rotation of the eyes during the animation builds a reference for results.

The algorithm uses the pictures from the model to generate pupil data. A partner work uses the pupil data to calculate a gaze direction. It is now possible to determine measurement uncertainty with the known rotation of the model as reference.