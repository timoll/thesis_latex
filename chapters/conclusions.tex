\chapter{Conclusion}
\label{chap:conclusions}
The realistic 3D model of the human eye and GazelleGlasses lays the foundation to test methods that calculate the gaze direction from images of the eyes. It is simple to perform additional test because no knowledge of the 3D modelling software is needed to create new test scenarios. The model is versatile as many different parameters can be configured. 

The algorithm to digitally detect the pupil is much more stable and accurate. One of the reasons for that is an ingenious method to eliminate detected edges that don't lie on the border between the pupil and the iris. Those edges would otherwise distort the result and make an accurate eye-tracking impossible.

A test with data from the model shows that a higher resolution improves the accuracy of the eye-tracking. 
